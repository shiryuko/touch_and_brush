%%%%%%%%%%%%%%%%%%%%%%%%%%%%%%%%%%%%%%%%%%%%%%%%%%%%%%%%%%%%%%%%%%%%%%%%
%Desarrollo de un juego para Nintendo DS | Trabajo de Fin de Grado
% Escuela Politécnica Superior de la Universidad de Alicante
% Realizado por: Carla Maciá Díez
% Contacto: carlamd1997@hotmail.com / cmd23@alu.ua.es
%%%%%%%%%%%%%%%%%%%%%%%%%%%%%%%%%%%%%%%%%%%%%%%%%%%%%%%%%%%%%%%%%%%%%%%%

\chapter{Resumen}
\textbf{Touch \& Brush} es un videojuego de agilidad mental para la consola \textbf{Nintendo DS}, desarollado en \textbf{C++} y el conjunto de herramientas de \textbf{devkitPro}, basado principalmente en \textbf{Magic Cat Academy} de Google (Doodle de Halloween 2016).

\vspace{0.5cm}

En cada \textbf{nivel}  el jugador deberá vencer a un número de enemigos para poder continuar y estos irán apareciendo alrededor de la pantalla e irán moviéndose hacia el jugador, el cual estará en el centro. Dichos enemigos poseerán un \textbf{patrón} visible que el jugador deberá \textbf{dibujar} correctamente en la pantalla táctil para poder derrotarlos antes de que ellos le alcancen y le quiten vida.

{\let\clearpage\relax\chapter*{Abstract}}
\textbf{Touch \& Brush} is a mental agility video game for the \textbf{Nintendo DS} console, developed in \textbf{C ++} and using \textbf{devkitPro} toolkit, based mainly on Google's \textbf{Magic Cat Academy} (2016 Halloween Doodle).

\vspace{0.5cm}

On each \textbf{level} the player must defeat a number of enemies in order to continue and they will appear around the screen and move towards the player, who will be in the center. These enemies will have a \textbf{pattern} that the player must correctly \textbf{draw} in the touch screen in order to defeat them before they reach  and damage him.


\chapter{Justificación y objetivos}
\thispagestyle{empty}
Desde que comencé el grado en ingeniería multimedia hace ya 5 años he podido participar en el desarrollo de un gran número de proyectos enfocados en los videojuegos que, aunque se traten de simples demos básicas con fines totalmente didácticos, me han servido para darme cuenta de que no solo me apasiona jugar a videojuegos, sino que también adoro crearlos. Es por eso que mi último proyecto como estudiante de este grado no podía ser otro que crear un videojuego para la consola que más horas de diversión me ha brindado en mi infancia: la Nintendo DS.

\vspace{0.5cm}

Este proyecto supone una gran ilusión para mi, pues ser capaz de crear un juego para una plataforma en la que años atrás yo disfrutaba y pasaba horas y horas jugando tanto a sus grandes títulos es un sentimiento que no puedo llegar a expresar con palabras, y que por supuesto, influirá a la hora de perfeccionar cada detalle. Además, para poder desarrollar un juego de una manera correcta debemos \textbf{entender cómo trabaja la máquina} a bajo nivel, así, siempre que cometamos un fallo seremos capaces de localizar el error más rápido y solucionarlo de una forma eficiente. En los ordenadores y consolas modernos esto puede llegar a ser un gran dolor de cabeza para el desarrollador, pues son \textbf{máquinas muy complejas}. Sin embargo, las \textbf{consolas retro}, al tratarse de una versión \textbf{simplificada}, son unas \textbf{excelentes candidatas} para comenzar desarrollando pequeños juegos que nos permitan entender su funcionamiento y que, al fin y al cabo, la base de las máquinas modernas también está ahí.

\vspace{0.5cm}

Por otro lado, la Nintendo DS es una consola con \textit{hardware} específico para el desarrollo de juegos a la que se le puede sacar muchísimo partido y, obviamente, las grandes empresas que antiguamente desarollaban para esta máquina no iban a andar compartiendo el código fuente ni enseñando a terceros. Es por ello que si como fanático deseas introducirte en el mundo del desarrollo para Nintendo DS y buscas información, lo más probable es que pases horas y horas leyendo artículos hechos por personas que han creado manuales con pequeñas demos y ejemplos pero que llevan muchos años sin actualizarse o no terminen de explicar del todo lo que tú esperas. Así pues, espero poder enfocar esta memoria a modo de \textbf{manual} para que le \textbf{sirva a cualquier persona que desee desarrollar para la Nintendo DS.}

\vspace{0.5cm}

Dicho esto, los objetivos a cumplir en la realización de este proyecto son:

\begin{itemize}
   \item \textbf{Analizar la Nintendo DS y entender sus capacidades y limitaciones.}
   \item  \textbf{Analizar librerías y frameworks existentes.}
    \item \textbf{Aprender/Profundizar programación en C++.}
    \item \textbf{Comprender el funcionamiento del hardware.}
     \item  \textbf{Diseñar y desarrollar un videojuego completo.}
     \item \textbf{Realizar la producción física del videojuego en cartucho.}
     \item \textbf{Elaborar un manual de desarrollo referenciable que aúne el conocimiento disponible sobre desarrollo en Nintendo DS.}
\end{itemize}


\cleardoublepage %salta a nueva página impar
\chapter{Agradecimientos}
Antes de nada, he de agradecer a todas las personas que me han ayudado, no solo durante todo el desarrollo de este proyecto, sino también durante toda mi etapa como estudiante de este grado. Ellos han sido gran parte de mi motivación a seguir adelante y mi apoyo en los momentos más difíciles.

\vspace{0.5cm}

En primer lugar gracias a mis padres, por proporcionarme los medios necesarios para poder formarme en un campo que tanto me gusta y por interesarse por éste a pesar de no entender mucho del tema. También, por saber detectar cuándo lo estaba pasando mal e intentar ayudarme al instante, a pesar de que yo sea de pocas palabras en ese aspecto.

\vspace{0.5cm}

A mi pareja y amigos, no solo a los antiguos, también a los que he conocido en el grado y que me han permitido aprender mucho de ellos. Ángel, Leire, Franck, Yaiza, Raquel, Mer, Alberto... con ellos y muchos más he podido compartir gran cantidad de trabajos, \textit{hobbies}, buenos y malos momentos que sin duda espero seguir compartiendo en el futuro.

\vspace{0.5cm}

Y por último, gracias a mi tutor y al profesorado del grado, en especial a los profesores de cuarto curso. Ellos me han ayudado a ver la luz al final del túnel y hacerme creer que puedo dedicarme a esto después de unos primeros años de caos y de estar perdida entre tantas asignaturas distintas. Gracias a su dedicación y esfuerzo los alumnos de multimedia hemos podido disfrutar al máximo nuestros ultimos años de carrera.

\vspace{0.5cm}

A todos y cada uno de vosotros, gracias.

\thispagestyle{empty}



%\cleardoublepage %salta a nueva página impar
%% Aquí va la dedicatoria si la hubiese. Si no, comentar la(s) linea(s) siguientes
%\chapter*{}
%\label{dedicatoria}
%\setlength{\leftmargin}{0.5\textwidth}
%\setlength{\parsep}{0cm}
%\addtolength{\topsep}{0.5cm}
%\begin{flushright}
%\small\em{
%Una pequeña dedicatoria al líder supremo de Corea del Norte.
%%A toda persona que ha pasado por mi vida,
%%sin las cuales no sería la persona que soy hoy.
%}
%\end{flushright}


\cleardoublepage %salta a nueva página impar
