%%%%%%%%%%%%%%%%%%%%%%%%%%%%%%%%%%%%%%%%%%%%%%%%%%%%%%%%%%%%%%%%%%%%%%%%
%Desarrollo de un juego para Nintendo DS | Trabajo de Fin de Grado
% Escuela Politécnica Superior de la Universidad de Alicante
% Realizado por: Carla Maciá Díez
% Contacto: carlamd1997@hotmail.com / cmd23@alu.ua.es
%%%%%%%%%%%%%%%%%%%%%%%%%%%%%%%%%%%%%%%%%%%%%%%%%%%%%%%%%%%%%%%%%%%%%%%%

\chapter{Terminología}

A  lo largo de este documento se van a utilizar una serie de abreviaturas con el fin de hacer la lectura más amena. Todos estos términos están explicados a continuación.

\begin{itemize}
  \item \textbf{NDS:} Nintendo DS.
  \item \textbf{NDSi:} Nintendo DSi.
  \item \textbf{GBA:} Game Boy Advance.
  \item \textbf{ARM9:} Procesador ARM946E-S de la Nintendo DS.
  \item \textbf{ARM7:} Procesador ARM7TDMI de la Nintendo DS.
  \item \textbf{R4:} Revolution for NDS.
  \item \textbf{RPG:} Role Playing Game o videojuego de rol. 
  \item \textbf{VRAM:} Memoria RAM de vídeo.
  \item \textbf{RAM:} Memoria de Acceso Aleatorio.
  \item \textbf{NPC:} Non Playable Characters, hace referencia a todos los personajes que aparecen en el juego que no son el jugador.
  \item \textbf{HUD:} Head-Up Display, es la información que, por lo general en un videojuego se muestra siempre por pantalla.
  \item \textbf{GRF:} Formato de archivo gráfico usado por Microsoft GraphEdit.
\end{itemize}
