%%%%%%%%%%%%%%%%%%%%%%%%%%%%%%%%%%%%%%%%%%%%%%%%%%%%%%%%%%%%%%%%%%%%%%%%
% Plantilla TFG/TFM
% Escuela Politécnica Superior de la Universidad de Alicante
% Realizado por: Jose Manuel Requena Plens
% Contacto: info@jmrplens.com / Telegram:@jmrplens
%%%%%%%%%%%%%%%%%%%%%%%%%%%%%%%%%%%%%%%%%%%%%%%%%%%%%%%%%%%%%%%%%%%%%%%%

\chapter{Introducción}

Este proyecto trata sobre el desarrollo de un \textbf{juego para la consola} de Nintendo, \textbf{Nintendo DS} y sus posteriores versiones (hasta la Nintendo DSi XL).

\vspace{0.5cm}

El videojuego, titulado Touch \& Brush, se trata de un juego de categoría \textbf{agilidad mental} y en el cual se progresa por \textbf{niveles}. En cada nivel, aparecerán una serie de \textbf{enemigos} que tendrán un \textbf{patrón específico} e irán a atacar al jugador. Este, si no quiere perder todas sus vidas y por tanto la partida, debe ser más rápido y \textbf{dibujar dichos patrones} en la pantalla táctil para derrotarlos.

\vspace{0.5cm}

Esta es una idea que puede parecer sencilla, pero que a nivel técnico lleva bastante trabajo, no solo por desarrollar todas las \textbf{mecánicas y aspectos visuales} para una consola limitada como es la Nintendo DS, sino también porque el propio proyecto requiere analizar, investigar, e incluso desarrollar un \textbf{algoritmo de reconocimiento de gestos}, para que el juego sea capaz de identificar el trazo del usuario. Y eso no es un trabajo que pueda hacerse en una tarde.

\vspace{0.5cm}

También, requiere trabajo tanto de \textbf{diseño del juego} completo, como \textbf{planificación} del proyecto, y sobre todo la parte de \textbf{documentación}, pues uno de los objetivos es que esta memoria sirva de referencia o manual a cualquiera con una base de programación que quiera desarrollar para NDS. Por ello, debemos plasmar cada concepto y procedimiento de la manera más detallada posible.

\vspace{0.5cm}

Por otro lado, la Nintendo DS, tal y como hemos dicho, se trata de una consola con limitaciones técnicas, es por ello que debemos \textbf{analizarla} previamente así como conocer las diferencias entre sus versiones. Tiene dos procesadores ARM que ejecutan los juegos originales y retrocompatibles, doble pantalla (siendo una de ellas táctil), zonas hardware dedicadas a los gráficos (VRAM, OAM), soporte para sonido estéreo...

\vspace{0.5cm}

Todos y cada uno de esos aspectos los describiremos en los apartados siguientes.























